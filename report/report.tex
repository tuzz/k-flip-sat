\documentclass{article}
\usepackage[a4paper, margin=0.7in]{geometry}
\usepackage[table]{xcolor}
\usepackage{pgfplotstable}
\pgfplotsset{compat=1.17}

\begin{document}
\title{A k-flip local search algorithm for SAT and MAX SAT}
\author{Chris Patuzzo}
\maketitle

\abstract
Local search can be applied to SAT by determining whether it is possible to
increase the number of satisfied clauses for a given truth assignment by
flipping at most $k$ variables. However, for a problem instance with $v$
variables, the search space is of order $v^k$. A naive approach that enumerates
every combination is impractical for all but the smallest of problems. This
paper outlines a hybrid approach that plays to the strength of modern SAT
solvers to search this space more efficiently. We describe an encoding of SAT
to a related problem –\linebreak k-Flip MAX SAT – and show how, through repeated
application, it can be used to solve SAT and MAX SAT problems. Finally, we test
the algorithm on a benchmark set with different values of $k$ to see how it
performs.

\section{Introduction}
- sat problems have hundreds or thousands of variables, doesn't scale

- explain k-flip max sat

- explain ipasir and justify it for this problem

\section{The encoding}

At a high-level, the encoding works by introducing a set of variables $A$ that
represents a hypothetical SAT solver's current truth assignment of variables
within some formula $F$. A corresponding set of variables $A'$ is introduced
that is allowed to differ by at most $k$ truth assignments from $A$.  We use a
counter circuit and a less-than comparator to enforce this constraint.

For each clause in $F$, we introduce a variable whose intended meaning is that
its related clause has not been satisfied by $A'$. Collectively, we call this
set $U$. We enforce that the number of true literals in $U$ is less than the
SAT solver's current number of unsatisfied clauses for $F$. We once again use a
counter circuit and less-than comparator to enforce this constraint.

\subsection{Flipped variables}
\subsection{Unsatisfied clauses}
\subsection{Parallel counter}
\subsection{Less-than comparator}

\section{Repeated application}

\section{Empirical results}

\pgfplotstableset{
  every head row/.style={after row=\hline},
  every first column/.style={column type/.add={}{|}},
  columns/x/.style={column name={}, postproc cell content/.code={}},
  color cells/.style={
    postproc cell content/.code={%
      \pgfkeysalso{@cell content=\rule{0cm}{2.4ex}%
      \pgfmathsetmacro\y{min(100,max(0,abs(round(##1 * 0.07))))}%
      \edef\temp{\noexpand\cellcolor{blue!\y}}\temp%
      \pgfmathtruncatemacro\x\y%
      \ifnum\x>50 \color{white}\fi%
      ##1}%
    },
  }
}

\newgeometry{left=0.5in}
\begin{table}
\centering
\pgfplotstabletypeset[color cells, font=\footnotesize]{
x 1 2 3 4 5 6 7 8 9 10 11 12 13 14 15 16 17 18 19 20
430 0 2 7 28 75 105 163 227 303 370 417 499 571 610 645 678 797 798 838 879
429 1 16 68 171 250 345 443 478 519 555 574 609 580 588 616 595 503 520 494 465
428 0 46 156 305 376 454 429 416 389 341 316 231 217 180 121 117 91 76 65 53
427 7 122 270 369 328 305 253 197 154 110 79 50 27 20 16 9 9 6 3 3
426 14 191 307 240 213 152 86 69 35 25 18 11 4 2 2 1 0 0 0 0
425 37 251 256 158 107 32 24 14 3 4 1 0 1 0 0 0 0 0 0 0
424 78 240 182 85 43 8 6 4 1 0 0 0 0 0 0 0 0 0 0 0
423 121 209 92 35 10 2 1 0 1 0 0 0 0 0 0 0 0 0 0 0
422 173 145 34 13 2 2 0 0 0 0 0 0 0 0 0 0 0 0 0 0
421 181 104 21 1 0 0 0 0 0 0 0 0 0 0 0 0 0 0 0 0
420 165 46 7 0 1 0 0 0 0 0 0 0 0 0 0 0 0 0 0 0
419 179 22 4 0 0 0 0 0 0 0 0 0 0 0 0 0 0 0 0 0
418 161 6 0 0 0 0 0 0 0 0 0 0 0 0 0 0 0 0 0 0
417 110 5 1 0 0 0 0 0 0 0 0 0 0 0 0 0 0 0 0 0
416 80 0 0 0 0 0 0 0 0 0 0 0 0 0 0 0 0 0 0 0
415 38 0 0 0 0 0 0 0 0 0 0 0 0 0 0 0 0 0 0 0
414 33 0 0 0 0 0 0 0 0 0 0 0 0 0 0 0 0 0 0 0
413 17 0 0 0 0 0 0 0 0 0 0 0 0 0 0 0 0 0 0 0
412 6 0 0 0 0 0 0 0 0 0 0 0 0 0 0 0 0 0 0 0
411 3 0 0 0 0 0 0 0 0 0 0 0 0 0 0 0 0 0 0 0
410 0 0 0 0 0 0 0 0 0 0 0 0 0 0 0 0 0 0 0 0
409 1 0 0 0 0 0 0 0 0 0 0 0 0 0 0 0 0 0 0 0
}
\caption{caption}
\end{table}
\restoregeometry
\end{document}
