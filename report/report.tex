\documentclass{article}
\usepackage[a4paper, margin=0.7in]{geometry}
\usepackage[table]{xcolor}
\usepackage{pgfplotstable}
\pgfplotsset{compat=1.17}

\begin{document}
\title{A k-flip local search algorithm for SAT and MAX SAT}
\author{Chris Patuzzo}
\maketitle

\abstract
Local search can be applied to SAT by determining whether it is possible to
increase the number of satisfied clauses for a given truth assignment by
flipping at most $k$ variables. However, for a problem instance with $v$
variables, the search space is of order $v^k$. A naive approach that enumerates
every combination is impractical for all but the smallest of problems. This
paper outlines a hybrid approach that plays to the strength of modern SAT
solvers to search this space more efficiently. We describe an encoding of SAT
to a related problem –\linebreak k-Flip MAX SAT – and show how through repeated
application, it can be used to solve SAT and MAX SAT problems. Finaly, we test
the algorithm on a benchmark set with different values of $k$ to see how it
performs.

\section{Introduction}
- sat problems have hundreds or thousands of variables, doesn't scale

- explain k-flip max sat

- explain ipasir and justify it for this problem

\section{The encoding}

At a high-level, the encoding works by introducing a set of variables $A$ that
represents a hypothetical SAT solver's current truth assignment of variables
within some formula $F$. A corresponding set of variables $A'$ is introduced
that is allowed to differ by at most $k$ truth assignments from $A$.  We use a
counter circuit and a less-than comparator to enforce this constraint.

For each clause in $F$, we introduce a variable whose intended meaning is that
the related clause has not been satisfied by $A'$. Collectively, we call this
set $U$. We enforce that the number of true literals in $U$ is less than the
SAT solver's current number of unsatisfied clauses for $F$. We once again use a
counter circuit and less-than comparator to enforce this constraint.

\subsection{Flipped variables}
\subsection{Unsatisfied clauses}
\subsection{Parallel counter}
\subsection{Less-than comparator}

\section{Repeated application}

\section{Empirical results}


\pgfplotstableset{
  every head row/.style={after row=\hline},
  every first column/.style={column type/.add={}{|}},
  color cells/.style={
    postproc cell content/.code={%
      \pgfkeysalso{@cell content=\rule{0cm}{2.4ex}\cellcolor{blue!##1}\pgfmathtruncatemacro\number{##1}##1}%
    },
    columns/x/.style={column name={}, postproc cell content/.code={}}
  }
}

\begin{table}
\centering
\pgfplotstabletypeset[color cells]{
x 1 2 3 4 5 6 7 8 9 10 11 12 13 14 15 16 17 18 19 20
430 0 0 0 1 4 4 8 10 17 17 17 22 21 28 24 32 42 31 41 37
429 0 0 1 10 16 15 20 22 17 25 26 21 27 26 30 22 14 21 18 22
428 0 3 11 13 14 14 16 17 14 12 13 11 10 6 6 6 4 7 1 1
427 0 7 11 18 14 23 8 9 10 5 4 6 2 0 0 0 0 1 0 0
426 1 6 15 9 9 3 7 2 1 1 0 0 0 0 0 0 0 0 0 0
425 3 9 10 7 2 1 0 0 1 0 0 0 0 0 0 0 0 0 0 0
424 4 13 5 2 1 0 1 0 0 0 0 0 0 0 0 0 0 0 0 0
423 5 10 5 3 0 0 0 0 0 0 0 0 0 0 0 0 0 0 0 0
422 7 7 0 1 0 0 0 0 0 0 0 0 0 0 0 0 0 0 0 0
421 3 4 4 0 0 0 0 0 0 0 0 0 0 0 0 0 0 0 0 0
420 9 4 2 0 0 0 0 0 0 0 0 0 0 0 0 0 0 0 0 0
419 6 1 1 0 0 0 0 0 0 0 0 0 0 0 0 0 0 0 0 0
418 8 1 0 0 0 0 0 0 0 0 0 0 0 0 0 0 0 0 0 0
417 11 0 0 0 0 0 0 0 0 0 0 0 0 0 0 0 0 0 0 0
416 4 0 0 0 0 0 0 0 0 0 0 0 0 0 0 0 0 0 0 0
415 1 0 0 0 0 0 0 0 0 0 0 0 0 0 0 0 0 0 0 0
414 2 0 0 0 0 0 0 0 0 0 0 0 0 0 0 0 0 0 0 0
413 1 0 0 0 0 0 0 0 0 0 0 0 0 0 0 0 0 0 0 0
}
\caption{caption}
\end{table}
\end{document}
