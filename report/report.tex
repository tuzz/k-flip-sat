\documentclass{article}
\usepackage[a4paper, margin=0.7in]{geometry}
\usepackage[table]{xcolor}
\usepackage[mathscr]{euscript}
\usepackage{pgfplotstable}
\usepackage{amsmath}
\usepackage{amsfonts}
\usepackage{multicol}
\pgfplotsset{compat=1.17}

\begin{document}
\title{A k-flip local search algorithm for SAT and MAX SAT}
\author{Chris Patuzzo}
\maketitle

\abstract
Local search can be applied to SAT by determining whether it is possible to
increase the number of satisfied clauses for a given truth assignment by
flipping at most $k$ variables. However, for a problem instance with $v$
variables, the search space is of order $v^k$. A naive approach that enumerates
every combination is impractical for all but the smallest of problems. This
paper outlines a hybrid approach that plays to the strength of modern SAT
solvers to search this space more efficiently. We describe an encoding of SAT
to a related problem –\linebreak k-Flip MAX SAT – and show how, through repeated
application, it can be used to solve SAT and MAX SAT problems. Finally, we test
the algorithm on a benchmark set with different values of $k$ to see how it
performs.

\section{Introduction}
- sat problems have hundreds or thousands of variables, doesn't scale

- explain k-flip max sat

- explain ipasir and justify it for this problem

\section{The encoding}

At a high level, the encoding takes some SAT formula $F$ and parameter $k$ and
transforms it into a new SAT formula $F'$ that is satisfiable if and only if
$F$ is satisfiable subject to two numerical constraints:

\begin{enumerate}
  \item The first numerical constraint enforces the ‘k-flips’ requirement. A
    set of variables $A$ is introduced that represents some truth assignment
    for $F$. A corresponding set of variables $A'$ is added that is allowed to
    differ by at most $k$ truth values from $A$. Intuitively, this delta is the
    subset of variables that has been ‘flipped’. We use a counter circuit and a
    less-than comparator to enforce this constraint.

  \item The second numerical constraint limits the number of unsatisfied
    clauses in $F$ subject to the set of truth values $A'$. For each clause in
    $F$, we introduce a variable whose intended meaning is that its related
    clause has not been satisfied by $A'$. Collectively, we call this set $U$.
    We once again use a counter circuit and less-than comparator to enforce
    that the number of true literals in $U$ is less than some value.
\end{enumerate}

\noindent Our encoding has the advantage of separating its numerical
constraints from their threshold values. The latter can either by specified by
appending unit clauses to $F'$ or through assumptions as part of the IPASIR
interface.

\subsection{Flipped variables}

Let $\#v$ be the number of variables in $F$. Add a clause to $F'$ that is
satisfied if either $A_i$ and $A'_i$ have the same truth value or $Fl_i$ is
true. Formula \ref{flipped2} is equivalent to Formula \ref{flipped1} but is
rewritten in conjunctive normal form.

\begin{equation}
  \label{flipped1}
  \bigwedge\limits_{i=1}^{\#v} A_i \to A'_i \lor Fl_i
\end{equation}\break

\begin{equation}
  \label{flipped2}
  \bigwedge\limits_{i=1}^{\#v} \neg{A_i} \lor A'_i \lor Fl_i
\end{equation}\break

\noindent The intended meaning of $Fl_i$ is that variable $i$ in $F$ has been
flipped from some pre-assigned truth value $A_i$ to a new value $A'_i$. However,
we do not add clauses that preclude $Fl_i$ from being true when $A_i$ and $A'_i$
are assigned the same value. In practice, it is never advantageous for a SAT
solver to do so due to the numeric constraints.

\subsection{Unsatisfied clauses}

Let $\#c$ be the number of clauses in $F$. Add a clause to $F'$ that is
satisfied if either clause $i$ in $F$ is satisfied or $U_i$ is true. Again, we
do not preclude $U_i$ from being true when clause $i$ is already satisfied.

\begin{equation}
  \label{unsat}
  \bigwedge\limits_{i=1}^{\#c} Clause_i \lor U_i
\end{equation}\break

\subsection{Parallel counter}

We encode two separate parallel counter circuits into $F'$. The first operates
on $Fl$ and the second on $U$. Since the method of encoding is the same, we
discuss it in general terms for a set $\mathscr{S}$. The objective of the
encoding is to introduce a set of variables $\mathscr{C}$ of size $\lceil
log_2(\mathscr{S}) \rceil$ such that the formula $F'$ is satisfiable if and
only if $\mathscr{C}$ is assigned truth values representing a binary number
equal to the count of true literals in $\mathscr{S}$.

The encoding works by first applying a half-adder gate to consecutive,
non-overlapping pairs of variables $a, b \in \mathscr{S}$. We use a propagation
complete encoding (Formula \ref{halfadder}) which can be derived from the
propagation complete encoding of a full-adder (Formula \ref{fulladder}) by
setting $carry_{in}$ to \textbf{false} and simplifying.

\begin{equation}
  \label{halfadder}
  \begin{split}
    a \lor \neg{b} \lor sum \\
    \neg{a} \lor \neg{b} \lor \neg{sum} \\
    \neg{a} \lor carry_{in} \lor sum \\
    a \lor \neg{carry_{in}} \lor \neg{sum} \\
    b \lor \neg{carry_{in}} \\
    a \lor b \lor \neg{sum} \\
  \end{split}
\end{equation}\break

\noindent The encoding then proceeds recursively. It subdivides the auxiliary
variables produced by the half-adders until either one or two pairs of variables
remain. If two pairs remain, a full-adder (Formula \ref{fulladder}) sums the
result. Afterwards a ripple-carry adder is used to recombine these sums. A
ripple-carry also makes use of multiple full-adders. Its description is omitted
here because it is encoded in a conventional way.

\begin{equation}
  \label{fulladder}
  \begin{split}
    a \lor \neg{b} \lor carry_{in} \lor sum \\
    a \lor b \lor \neg{carry_{in}} \lor sum \\
    \neg{a} \lor \neg{b} \lor carry_{in} \lor \neg{sum} \\
    \neg{a} \lor b \lor \neg{carry_{in}} \lor \neg{sum} \\
    \neg{a} \lor carry_{out} \lor sum \\
    a \lor \neg{carry_{out}}\lor \neg{sum} \\
    \neg{b} \lor \neg{carry_{in}} \lor carry_{out} \\
    b \lor carry_{in} \lor \neg{carry_{out}} \\
    \neg{a} \lor \neg{b} \lor \neg{carry_{in}} \lor sum \\
    a \lor b \lor carry_{in} \lor \neg{sum} \\
  \end{split}
\end{equation}\break

\noindent In general, when two N-bit binary numbers are summed, this can result
in an (N+1)-bit binary number. However, since we know the sum will not exceed
$\vert \mathscr{S} \vert$, there is no need to introduce redundant auxiliary
variables that would always be false. This is a small optimisation that also
helps the SAT solver reject assignments that would inevitably lead to conflict
when the less-than clauses are considered.

\subsection{Less-than comparator}

The less-than comparator makes use of three logical operators: \textsc{and},
\textsc{or} and \textsc{eq}. We use the Tseitin encodings of these gates as
shown in Formulas \ref{and}, \ref{or} and \ref{eq} respectively.

\begin{multicols}{3}
  \begin{equation}
    \label{and}
    \begin{split}
      \neg{a} \lor \neg{b} \lor out \\
      a \lor \neg{out} \\
      b \lor \neg{out} \\
    \end{split}
  \end{equation}\break

  \begin{equation}
    \label{or}
    \begin{split}
      a \lor b \lor \neg{out} \\
      \neg{a} \lor out \\
      \neg{b} \lor out \\
    \end{split}
  \end{equation}\break

  \begin{equation}
    \label{eq}
    \begin{split}
      \neg{a} \lor \neg{b} \lor out \\
      a \lor b \lor out \\
      a \lor \neg{b} \lor \neg{out} \\
      \neg{a} \lor b \lor \neg{out} \\
    \end{split}
  \end{equation}\break
\end{multicols}

\noindent First, we define a new operator that takes two variables and sets
$out$ to true when $a$ is strictly less than $b$.

\begin{equation}
  \textsc{lt}(a, b) = \textsc{and}(-a, b)
\end{equation}\break

\noindent We then define a recursive operator for two sets of variables $A$,
$B$ and $i \in \mathbb{Z}^*$.

\begin{equation}
  \textsc{lt}^*(A, B, i) = \begin{cases}
    \textsc{lt}(A_i, B_i) & i = 0 \\
    \textsc{or}(\textsc{lt}(A_i, B_i), \textsc{and}(\textsc{equal}(A_i, B_i), \textsc{lt}^*(A, B, i - 1))) & i > 0
  \end{cases}
\end{equation}\break

\noindent The $\textsc{lt}^*$ operator tests whether variable $A_i$ is strictly
less than $B_i$. If it is, the output of the operator is true. Otherwise, if
they are equal, it recursively tests the $i - 1$th bit until $i$ reaches $0$. We
use the convention that index $0$ is the least-significant bit and index $\vert
A \vert - 1$ is the most-significant bit of a binary number.

Finally, we encode the constraint that the binary number represented by
$\mathscr{C}$ is less than some threshold value $\mathscr{T}$ such that  $\vert
\mathscr{T} \vert = \vert \mathscr{C} \vert$ by conjuncting clauses generated
by the $\textsc{lt}^*$ operator.

\begin{equation}
  \bigwedge \textsc{lt}^*(\mathscr{C}, \mathscr{T})
\end{equation}\break

\section{The algorithm}

\section{Empirical results}

\pgfplotstableset{
  every head row/.style={after row=\hline},
  every first column/.style={column type/.add={}{|}},
  columns/x/.style={column name={}, postproc cell content/.code={}},
  color cells/.style={
    postproc cell content/.code={%
      \pgfkeysalso{@cell content=\rule{0cm}{2.4ex}%
      \pgfmathsetmacro\y{min(100,max(0,abs(round(##1 * 0.04))))}%
      \edef\temp{\noexpand\cellcolor{blue!\y}}\temp%
      \pgfmathtruncatemacro\x\y%
      \ifnum\x>50 \color{white}\fi%
      ##1}%
    },
  }
}

\newgeometry{left=0.4in}
\begin{table}
\centering
\pgfplotstabletypeset[color cells, font=\footnotesize]{
  x 1 2 3 4 5 6 7 8 9 10 11 12 13 14 15 16 17 18 19 20
430 0 4 24 49 117 193 283 389 484 635 739 843 985 1058 1154 1214 1343 1366 1461 1534
429 3 27 112 281 429 580 756 834 913 964 979 1062 1026 1035 1019 974 914 937 861 789
428 1 87 294 522 660 798 754 741 704 597 570 424 381 315 249 243 174 139 122 118
427 12 202 462 641 621 543 459 345 284 201 139 109 52 38 26 18 19 7 6 9
426 20 319 535 457 360 258 157 111 63 52 26 17 8 4 2 1 0 1 0 0
425 66 448 463 269 182 61 34 29 5 6 2 0 1 0 0 0 0 0 0 0
424 129 435 311 148 66 18 11 6 1 0 0 0 0 0 0 0 0 0 0 0
423 215 379 143 59 13 2 1 0 1 0 0 0 0 0 0 0 0 0 0 0
422 295 253 53 25 4 2 0 0 0 0 0 0 0 0 0 0 0 0 0 0
421 328 168 36 3 2 0 0 0 0 0 0 0 0 0 0 0 0 0 0 0
420 308 80 17 1 1 0 0 0 0 0 0 0 0 0 0 0 0 0 0 0
419 312 38 4 0 0 0 0 0 0 0 0 0 0 0 0 0 0 0 0 0
418 257 7 0 0 0 0 0 0 0 0 0 0 0 0 0 0 0 0 0 0
417 211 8 1 0 0 0 0 0 0 0 0 0 0 0 0 0 0 0 0 0
416 135 0 0 0 0 0 0 0 0 0 0 0 0 0 0 0 0 0 0 0
415 71 0 0 0 0 0 0 0 0 0 0 0 0 0 0 0 0 0 0 0
414 52 0 0 0 0 0 0 0 0 0 0 0 0 0 0 0 0 0 0 0
413 25 0 0 0 0 0 0 0 0 0 0 0 0 0 0 0 0 0 0 0
412 8 0 0 0 0 0 0 0 0 0 0 0 0 0 0 0 0 0 0 0
411 6 0 0 0 0 0 0 0 0 0 0 0 0 0 0 0 0 0 0 0
410 0 0 0 0 0 0 0 0 0 0 0 0 0 0 0 0 0 0 0 0
409 1 0 0 0 0 0 0 0 0 0 0 0 0 0 0 0 0 0 0 0
}
\caption{caption}
\end{table}
\restoregeometry
\end{document}
